\documentclass{article}
\usepackage{fancyhdr}
\usepackage{graphicx, wrapfig, subcaption, setspace, booktabs}
\usepackage[english]{babel}
\usepackage[utf8]{inputenc}

\usepackage{comment}
\usepackage{listings}
\usepackage{xcolor}
\usepackage{textcomp}
\graphicspath{ {./image/} }



%New colors defined below
\definecolor{codegreen}{rgb}{0,0.6,0}
\definecolor{codegray}{rgb}{0.5,0.5,0.5}
\definecolor{codepurple}{rgb}{0.58,0,0.82}
\definecolor{backcolour}{rgb}{0.95,0.95,0.92}

%Code listing style named "mystyle"
\lstdefinestyle{mystyle}{
	backgroundcolor=\color{backcolour},
	commentstyle=\color{codegreen},
	keywordstyle=\color{magenta},
	numberstyle=\tiny\color{codegray},
	stringstyle=\color{codepurple},
	basicstyle=\ttfamily\footnotesize,
	breakatwhitespace=false,
	breaklines=false,
	captionpos=b,
	keepspaces=true,
	numbers=left,
	numbersep=3pt,
	showspaces=false,
	showstringspaces=false,
	showtabs=false,
	tabsize=4
}

%"mystyle" code listing set
\lstset{style=mystyle}


\newcommand{\HRule}[1]{\rule{\linewidth}{#1}}
\onehalfspacing
\setcounter{tocdepth}{5}
\setcounter{secnumdepth}{5}

%-------------------------------------------------------------------------------
% HEADER & FOOTER
%-------------------------------------------------------------------------------
\pagestyle{fancy}
\fancyhf{}
\setlength\headheight{15pt}
\fancyhead[L]{Assignment Report}
\fancyhead[R]{Ebru Kardas}

\begin{comment}
https://tr.overleaf.com/latex/examples/fysik-template/hdpmbqvfqpsw
-Udledninger
$$
\begin{aligned}


\end{aligned}
$$

-Opgavetekst
\begin{figure}[H]
\includegraphics[width=0.5\textwidth]{"path"}
\end{figure} 


-Opgave billede med tekst
\begin{figure}[H]
\caption{"Billedtekst"}
\includegraphics[width=0.5\textwidth]{"path"}
\end{figure} 

-Værdier
$\\

$
\end{comment}


\begin{document}

\title{ \normalsize \textsc{ }
		\\ [4.0cm]
		\HRule{5pt} \\
		\LARGE \textbf{\uppercase{Assignment Report \\ Game: Battleship}}
		\HRule{5pt} \\ [2.5cm]
		\normalsize \vspace*{5\baselineskip}}

\author{Ebru Kardas}

\maketitle

%%%%%%%%%%%%%%%%%%%%%%%%%%%%%%%%%%% PROBLEM %%%%%%%%%%%%%%%%%%%%%%%%%%%%%%%%%%%
\newpage
\section{Problem}

	Design and implement classical "Battleships" game. "Battleships" is a two player game which 
	is played on a 10x10 grid board.
	
	\subsection{Ships}
	There are 4 different types of ships in the game. These ships are submarine (S), destroyer (D),
	cruiser (C) and battleship (B). They can be placed on the grid-based game board horizontally or
	vertically. The only difference between these ships is their sizes. They cover 1, 2, 3 and 4 squares on
	the grid board respectively. Any square of a ship can take a hit. If all squares of a ship are hit, then
	the ship is sank.

	\subsection{Players}
	Each player in the game has 10 ships containing 4 submarines, 3 destroyers, 2 cruisers and a
	battleship. Players place their ships on their own 10x10 grid game board, and attack to each other in
	turn. To attack the opponent, player says a game board coordinate such as 'H7' which states the 7th
	column of the 8th (H) row. If there is a ship piece in that coordinate, the ship is hit. If all pieces of
	the ship is hit, then it is sank. The opponent should warn the other player about the success or fail of
	the attack. Player whose all ships are sank loses the game.
	
	
	
%%%%%%%%%%%%%%%%%%%%%%%%%%%%%%%%%%% SOLUTION %%%%%%%%%%%%%%%%%%%%%%%%%%%%%%%%%%
\newpage
\section{My Design For The Problem: Solution}

Project is build on IntelliJ IDEA 2021.1.1 (Community Edition) with openjdk-16 (version 16.0.1).

	\subsection{Input Checks}
	Taken input for coordinate has to contain one letter, upper or lower-case, and a number. This input shows
	the coordinate of the board. Rows are represented with letter (A-J) and columns are represented with numbers (0-9).

	On start state, after coordinate the orientation of the ship is taken as character. The orientation is 
	represented with 'v' character for vertical, 'h' character for horizontal.	

	\subsection{Ship Design}
	The abstract "Ship" class is extended by ship types. 
	The class itself and the get method is abstract. Get method is implemented and for each ship size, 
	the constructor of the super class is used.

	The "Ship" class encapsulates all operations of the ships. 

	\subsection{Player Design with State Design Pattern}

	State design pattern is used for players. 
	There are three state for each player:
	\begin{itemize}
		\item Start State
		\item Take Turn State
		\item Finish State
	\end{itemize}

		\newpage
		\subsubsection{Start State}
		If the player is on "Start State", all information of the ships (4 submarines, 3 destroyers, 2 cruisers
		and 1 battleship) are entered from console as described. 

		First player A (or player 1) enters the inputs. After player B (player 2) enters.

		All inputs are checked on this stage.

		\subsubsection{Take Turn State}
		First of all, player A starts the game. Empty table is shown to the player. 
		Empty parts are shown as " - ".\newline
		Selected but not the part of the ship parts are shown as " S "..\newline
		Hit parts of a ship are shown as " H ".\newline

		The player enters a coordinate, which is also checked whether it is valid.
		Player is informed if the hit is successful or not.

		\subsubsection{Finish State}
		If the player is on "Finish State", it is checked if the user win. If the player is win, game is over.


\end{document}